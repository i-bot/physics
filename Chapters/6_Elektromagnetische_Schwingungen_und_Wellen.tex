\section{Elektromagnetische Schwingungen und Wellen}

\subsection{Elektrischer Schwingkreis (Kondensator $K$, Kapazität $C$ von $K$, Spule $S$,
Induktivität $L$ von $S$, Schwingungsdauer $T$, Frequenz $f$, Winkelgeschwindigkeit $w$)}

\leftskip1em
\paragraph{Thomsongleichung}

\begin{align*}
  T &= 2 \pi \cdot \sqrt{L \cdot C} &\\
  [T] &= \sqrt{\frac{Vs \cdot As}{A \cdot V}} = s &\\
  &\\
  f &= \frac{1}{T} = \frac{1}{2 \pi \cdot \sqrt{L \cdot C}} &\\
  w &= 2 \pi \cdot f = \frac{1}{\sqrt{L \cdot C}} &\\
\end{align*}

\paragraph{Zeitlicher Ablauf (maximal Spannung $U_0$ in $K$, maximale Stromstärke $I_0$ in $S$,
Zeitpunkt $t$)}

\subparagraph{Spannung $U_K$ in $K$}

\begin{align*}
  U_C \left(t\right) &= U_0 \cdot \cos \left(w \cdot t\right) &\\
\end{align*}

\subparagraph{Stromstärke $I_S$ in $S$}

\begin{align*}
  I_S \left(t\right) &= - I_0 \cdot \sin \left(w \cdot t\right) &\\
\end{align*}

\paragraph{Energie $E_0$ im Schwingkreis (Energie $E_{el}$ in $K$, Energie $E_{mag}$
im $S$, Zeitpunkt $t$)}

\begin{align*}
  E_0 &= E_{el} \left(t\right) + E_{mag} \left(t\right) &\\
\end{align*}

\subparagraph{Energie im Kondensatorfeld}

\begin{align*}
  E_{el} &= \frac{1}{2} \cdot C U^2 &\\
  E_{el} \left(t\right) &= \frac{1}{2} \cdot C \cdot U_0 \cdot \cos^2 \left(w \cdot t\right) &\\
\end{align*}

\subparagraph{Energie im Magnetfeld}

\begin{align*}
  E_{mag} &= \frac{1}{2} \cdot L I^2 &\\
  E_{mag} \left(t\right) &= \frac{1}{2} \cdot L \cdot I_0 \cdot - \sin^2 \left(w \cdot t\right) &\\
\end{align*}

\leftskip0em

\subsection{Dipolstab $S$ (Länge $l$ von $S$, Schwingungsfrequenz $f$ von $S$,
Wellenlänge $\lambda$, Ausbreitungsgeschwindigkeit $c$)}

\begin{align*}
  l &= \frac{1}{2} \cdot \lambda &\\
  c &= \lambda \cdot f
\end{align*}
