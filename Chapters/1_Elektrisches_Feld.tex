\section{Elektrisches Feld}

\subsection{Plattenkondensator $K$ (Spannung $U$ von $K$, Probeladung $Q$)}

\leftskip1em
\paragraph{Elektrische Feldstärke $E$ (Plattenabstand $d$ von $K$, Kraft $F$ auf $Q$)}

\begin{align*}
  E &= \frac{F}{Q} = \frac{U}{d} &\\
  [E] &= \frac{N}{C} = \frac{Nm}{C \cdot m} = \frac{VAs}{As \cdot m} = \frac{V}{m} &\\
\end{align*}

\paragraph{Elektrische Arbeit $W_{el}$ (Strecke $s$, Kraft $F$ auf $Q$)}

\begin{align*}
  W_{el} &= F \cdot s = Q \cdot E \cdot s &\\
  [W_{el}] &= Nm = VAs &\\
\end{align*}

\paragraph{Elektrisches Potenzial $\gamma$ (Plattenabstand $d$ von $K$, Strecke $s$)}

\begin{align*}
  \gamma &= \frac{W_{el}}{Q} = \frac{Q \cdot E \cdot s}{Q} = E \cdot s = \frac{U}{d} \cdot s &\\
  [\gamma] &= \frac{VAs}{As} = V &\\
\end{align*}

\paragraph{Kapazität $C$ (Plattenabstand $d$ von $K$, Plattenfläche $A$ von $K$)}

\begin{align*}
  C &= \epsilon_0 \epsilon_r \cdot \frac{A}{d} = \frac{Q}{U} &\\
  [C] &= \frac{C}{Vm} \cdot \frac{m²}{m} = \frac{C}{V} = F(arad) &\\
\end{align*}

\paragraph{Energiegehalt, Aufladarbeit $W_K$ von $K$}

\begin{align*}
  W_{K} &= \frac{1}{2}QU = \frac{1}{2}CU² &\\
  [W_{K}] &= \frac{C}{V} \cdot V² = VAs &\\
\end{align*}

\leftskip0em

\subsection{punktförmige Ladung $Q_1$ (Probeladung $Q_2$)}

\leftskip1em
\paragraph{Elektrische Feldstärke $E$ (Abstand $r$ zwischen $Q_1$ und $Q_2$)}

\begin{align*}
  E &= \frac{\left|Q_1\right|}{A \cdot \epsilon_0} = \frac{\left|Q_1\right|}{4 \pi r² \cdot
  \epsilon_0} &\\
  [E] &= \frac{C}{m² \cdot \frac{C}{Vm}} = \frac{V}{m} &\\
\end{align*}

\paragraph{Coulomb-Gesetz $F$ (Abstand $r$ zwischen $Q_1$ und $Q_2$)}
\begin{align*}
  F &= E \cdot Q_2 = \frac{\left| Q_1 \cdot Q_2 \right|}{4 \pi r² \cdot \epsilon_0} &\\
  [F] &= \frac{V \cdot As}{m} = \frac{Nm}{m} = N &\\
\end{align*}

\paragraph{Elektrisches Potenzial $\gamma$ (Abstand $r = s$ zwischen $Q_1$ und $Q_2$, $
\gamma(\infty) = 0$)}

\begin{align*}
  \gamma &= E \cdot s = \frac{\left|Q_1\right|}{4 \pi r² \cdot \epsilon_0} \cdot r =
  \frac{\left|Q_1\right|}{4 \pi r \cdot \epsilon_0} &\\
  [\gamma] &= \frac{V}{m} \cdot m = V &\\
\end{align*}

\leftskip0em
