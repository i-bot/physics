\section{Bewegung geladener Teilchen in elektrischen | magnetischen Feldern}

\subsection{Bewegungsgrundlagen}

\leftskip1em
\paragraph{Bewegung in Richtung der Feldkraft (Beschleunigung $a$, Zeit $t$,
Anfangsgeschwindigkeit $v_0$, Anfangsposition $x_0$)}
\begin{align*}
  v(t) &= v_0 + a \cdot t &\\
  x(t) &= x_0 + v_0 \cdot t + \frac{1}{2}a \cdot t² &\\
\end{align*}

\paragraph{Bewegung entgegen der Feldkraft (Beschleunigung $a$, Zeit $t$,
Anfangsgeschwindigkeit $v_0$, Anfangsposition $x_0$)}
\begin{align*}
  v(t) &= v_0 - a \cdot t &\\
  x(t) &= x_0 + v_0 \cdot t - \frac{1}{2}a \cdot t² &\\
\end{align*}

\paragraph{2.Newtonsche Gesetz (Kraft $F$, Beschleunigung $a$, Masse $m$)}
\begin{align*}
  F &= m \cdot a &\\
\end{align*}

\paragraph{Beschleunigung $a$, Geschwindigkeitsdifferenz $\delta v$, zurückgelegter Weg $\delta x$}
\begin{align*}
  \left( \Delta v \right) ^2 &= 2a \cdot \Delta x &\\
\end{align*}

\paragraph{Zentralkraft $F_Z$ (Masse $m$, Geschwindigkeit $v$, Kreisradius $r$)}
\begin{align*}
  F_Z &= \frac{m \cdot v^2}{r} &\\
\end{align*}

\leftskip0em

\subsection{Bewegung im elektrischen Feld eines Plattenkondensators $K$ (Probeladung $Q$)}

\leftskip1em
\paragraph{Längsfeld (elektrische Feldstärke $E$ zwischen $K$, Spannung $U$ von $K$,
Plattenabstand $d$ von $K$, Kraft $F$ auf $Q$, Masse $m$ von $Q$, Endgeschwindigkeit $v$ von $Q$)}

\subparagraph{Beschleunigung}
\begin{align*}
  a &= \frac{F}{m} = \frac{Q \cdot E}{m} = \frac{Q \cdot U}{m \cdot d}
\end{align*}

\subparagraph{Beschleunigungsarbeit}
\begin{align*}
  W &= F \cdot d = Q \cdot E \cdot d = Q \cdot \frac{U}{d} \cdot d = Q \cdot U
\end{align*}

\subparagraph{Kinetische Energie nach $d$}
\begin{align*}
  E_{kin} &= W &\\
  \frac{1}{2} m \cdot v² &= Q \cdot U &\\
\end{align*}

\paragraph{Querfeld $v_0 \perp E \perp F$ (elektrische Feldstärke $E$ zwischen $K$, Spannung $U$ von
$K$, Plattenabstand $d$ von $K$, Plattenlänge $l$ von $K$, Anfangsgeschwindigkeit $v_0$ von $Q$,
Kraft $F$ auf $Q$, Masse $m$ von $Q$, Endgeschwindigkeit $v$ von $Q$)}

\subparagraph{Bewegung in X-Richtung}
\begin{align*}
  l &= v_0 \cdot t
\end{align*}

\subparagraph{Bewegung in Y-Richtung}
\begin{align*}
  a_y &= \frac{F}{m} = \frac{Q \cdot E}{m} = \frac{Q \cdot U}{m \cdot d} &\\
  v_y \left( t \right) &= a_y \cdot t &\\
  y \left( t \right) &= \frac{1}{2}a_y \cdot t² &\\
  y \left( x \right) &= \frac{1}{2}a_y \cdot \left( \frac{x}{v_0}\right)^2 = \frac{a_y \cdot x²}{2
  \cdot v_0^2}
\end{align*}

\subparagraph{Resultierende Bewegung}
\begin{align*}
  v_{ges} \left( t \right) &= \sqrt{v_0 ^ 2 + v_y \left( t \right) ^ 2} &\\
  \tan \alpha &= \frac{v_y \left( t \right)}{v_0} &\\
\end{align*}

\leftskip0em

\subsection{Bewegung im Magnetfeld $B$ (Probeladung $Q$, Teilchengeschwindigkeit $v_Q$ von $Q$,
Teilchenmasse $m$ von $Q$)}

\leftskip1em
\paragraph{Wirksame Magnetische Flussdichte $B_w$ (Winkel $\alpha$)}
\begin{align*}
  B_w &= B \cdot \sin \alpha &\\
\end{align*}

\paragraph{Lorentzkraft $F_{mag}$ (gerader Probeleiter $P$, Länge $l$ von
$P$, Stromstärke $I$ von $P$)}
\begin{align*}
  I &= \frac{Q}{t} &\\
  l &= \frac{v_Q}{t} &\\
  F_{mag} &= l \cdot I \cdot B = v_Q \cdot Q \cdot &\\
\end{align*}

\paragraph{Kreisbahn von $Q$ in $B$ ($v_Q \perp B$, Kreisradius $r$)}
\begin{align*}
  F_{mag} &= F_Z &\\
  Q \cdot v_Q \cdot B &= \frac{m \cdot v_Q^2}{r} &\\
  Q \cdot B &= \frac{m \cdot v_Q}{r} &\\
\end{align*}

\paragraph{Spezifische Ladung $\frac{Q}{m}$ von $Q$ (Beschleunigungsspanngung $U$)}
\begin{align*}
  W_{el} &= E_{kin} &\\
  Q \cdot U &= \frac{1}{2}m \cdot v_Q^2 &\\
  Q \cdot U &= \frac{1}{2}m \cdot \left( Q \cdot B \cdot \frac{r}{m} \right)^2 &\\
  \frac{Q}{m} &= \frac{2 \cdot U}{B^2 \cdot r^2} &\\
\end{align*}

\paragraph{Massenspektrometer (Filtermagnetfeld $B_{Filter}$, Plattenkondensator $K$, Elektrisches
Feld $E$ von $K$, $B_{Filter} \perp E$, Kreisradius $r$)}

\subparagraph{Geschwindigkeitsfilter}
\begin{align*}
  F_{mag} &= F_{el} &\\
  Q \cdot v_Q \cdot B_{Filter} &= Q \cdot E &\\
  v_Q &= \frac{E}{B_{Filter}} &\\
\end{align*}

\subparagraph{Detektor}
\begin{align*}
  F_{mag} &= F_Z &\\
  Q \cdot B &= \frac{m \cdot v_Q}{r} &\\
  \frac{Q}{m} &= \frac{E}{r \cdot B \cdot B_{Filter}} &\\
\end{align*}

\paragraph{Hall-Effekt (Probeleiter $P$, Volumen $V$ von $P$, Länge $l$ von $P$, Breite $b$ von $P$,
Dicke $d$ von $P$, Driftgeschwindigkeit $v_{Drift}$ von $P$, Stromstärke $I$ von $P$, Elektrisches
Feld $E_{Hall}$, Hall-Spannung $U_{Hall}$, $E \perp B$, Elektronenzahl $N$, Ladungsträgerdichte $n$)}
\begin{align*}
  V &= l \cdot b \cdot d &\\
  \frac{1}{n} &= \frac{V}{N}
\end{align*}

\begin{align*}
  \Delta t &= \frac{Q}{I} = \frac{N \cdot e}{I} &\\
  v_{Drift} &= \frac{l}{\Delta t} = \frac{l \cdot I}{N \cdot e}
\end{align*}

\begin{align*}
  F_{el} &= F_{mag} &\\
  \frac{U_{Hall}}{b} &= v_{Drift} \cdot B &\\
  U_{Hall} &= v_{Drift} \cdot B \cdot b = \frac{I \cdot l}{N \cdot e} \cdot B \cdot b = &\\
  &= \frac{I \cdot l}{N \cdot e} \cdot B \cdot b \cdot \frac{d}{d} = \frac{I \cdot B}{N \cdot e}
  \cdot \frac{V}{d} = &\\
  &= \frac{1}{n \cdot e} \cdot \frac{I \cdot B}{d} &\\
\end{align*}

\leftskip0em
